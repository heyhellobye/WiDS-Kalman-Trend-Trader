\documentclass[11pt]{article}

\usepackage[a4paper,margin=1in]{geometry}
\usepackage{amsmath,amssymb}
\usepackage{setspace}
\usepackage{graphicx}
\usepackage{booktabs}

\setstretch{1.1}

\title{\textbf{WiDS Kalman Filtered Trend Trader}\\
Assignment 3}
\author{Lavanya Padole 24b1292}

\begin{document}
\maketitle

% ======================================================
\section{Task 1: Environment Setup}
% ======================================================

The \texttt{FrozenLake-v1} environment from the \texttt{gymnasium} library was successfully
initialized. The environment contains a discrete state space with 16 states corresponding
to positions on a $4 \times 4$ grid and a discrete action space with 4 possible actions:
left, down, right, and up.

Upon resetting the environment, a valid initial state was returned, confirming correct
installation and functioning of the reinforcement learning environment. This verified that
the setup was ready for subsequent reinforcement learning experiments.

% ======================================================
\section{Task 2: Tabular Q-Learning on FrozenLake}
% ======================================================

Tabular Q-learning was implemented to solve the \texttt{FrozenLake-v1} environment using an
$\epsilon$-greedy exploration policy.

\subsection{Final Hyperparameters}

The following hyperparameters were used for training:

\begin{itemize}
\item Learning rate ($\alpha$): 0.1
\item Discount factor ($\gamma$): 0.99
\item Initial exploration rate ($\epsilon$): 1.0
\item Minimum exploration rate: 0.01
\item Exploration decay rate: 0.999
\item Number of episodes: 20,000
\end{itemize}

\subsection{Results and Observations}

At the beginning of training, the agent exhibited near-zero success due to random exploration.
As training progressed and the exploration rate decayed, the agent gradually learned a policy
that increased the probability of reaching the goal state.

After training, the exploration rate converged to the minimum value of 0.01, and the agent
achieved an average success rate of approximately 68\% over the final episodes. Perfect
performance was not achieved due to the stochastic nature of the FrozenLake environment, where
state transitions are probabilistic even under optimal action selection.

These results demonstrate that tabular Q-learning is capable of learning a meaningful policy
in environments with discrete state and action spaces, though performance is bounded by
environmental randomness.

% ======================================================
\section{Task 3: Deep Q-Learning for MountainCar}
% ======================================================

\subsection{Limitations of Tabular Q-Learning}

The \texttt{MountainCar-v0} environment has a continuous state space consisting of position
and velocity. Tabular Q-learning is not well-suited for such environments, as discretizing
the state space leads to a rapid increase in the size of the Q-table and poor generalization.

\subsection{Deep Q-Network (DQN) Approach}

To address this limitation, a Deep Q-Network (DQN) was implemented. In this approach, a neural
network is used to approximate the Q-function, enabling the agent to generalize across
continuous state spaces.

The DQN architecture consists of fully connected layers with ReLU activations and outputs
Q-values corresponding to each discrete action. Experience replay is used to decorrelate
training samples, and a target network is employed to stabilize training.

\subsection{Training Behavior}

During training, the agent initially selects actions randomly due to high exploration.
As training progresses and the exploration rate decays, the agent begins to learn strategies
that allow it to build momentum and climb the slope toward the goal state.

Although the agent does not achieve the goal in early episodes, gradual improvements in
episode reward indicate learning progress. This behavior is consistent with typical DQN
training dynamics in the MountainCar environment, where learning is incremental and requires
many episodes to converge.

\subsection{Discussion}

The DQN-based approach demonstrates how neural networks enable reinforcement learning agents
to handle continuous state spaces more effectively than tabular methods. While convergence
may be slow and sensitive to hyperparameter choices, DQN provides a scalable solution for
complex environments that are infeasible for traditional tabular Q-learning.

% ======================================================
\section{Conclusion}
% ======================================================

In this assignment, reinforcement learning techniques were applied to progressively more
challenging environments. Tabular Q-learning proved effective for discrete environments such
as FrozenLake, while Deep Q-Learning was shown to be a more suitable approach for continuous
state spaces like MountainCar.

These experiments highlight the importance of selecting appropriate learning algorithms based
on environment characteristics and demonstrate the strengths and limitations of both tabular
and deep reinforcement learning methods.

\end{document}
