\documentclass[11pt]{article}

\usepackage[a4paper,margin=1in]{geometry}
\usepackage{amsmath}
\usepackage{graphicx}
\usepackage{hyperref}
\usepackage{setspace}

\setstretch{1.1}

\title{\textbf{WiDS Kalman Filtered Trend Trader}\\
Multi-Asset Portfolio Management}
\author{Lavanya Padole 24B1292}

\begin{document}
\maketitle

\section{Objective}

The objective of this project is to extend a Kalman-filtered trend following strategy from a single asset to a multi-asset portfolio management framework.
Instead of making trading decisions for one asset independently, the system dynamically allocates capital across multiple assets using Kalman-filtered trend information while maintaining strict causality and realistic backtesting assumptions.

The primary goal is to move from single-asset trading to portfolio-level decision making using interpretable, rule-based logic.

\section{Asset Universe and Data}

The portfolio consists of the following assets:
\begin{itemize}
    \item Bitcoin (BTC): high volatility, high risk asset
    \item Nifty 50: Indian equity market benchmark
    \item Gold: defensive and hedge asset
    \item Cash: risk-free holding
\end{itemize}

Daily price data from 2015 to 2024 is used for all assets.
Data is obtained from Yahoo Finance using adjusted prices to ensure consistency.
All asset time series are aligned on common timestamps, missing values are removed, and preprocessing is performed independently for each asset to avoid information leakage.

Cash is modeled implicitly as unallocated capital, allowing the strategy to actively reduce market exposure during periods of unfavorable trends.

\section{Kalman Filtering Methodology}

A Kalman filter is applied independently to each asset's price series to extract latent trend information.
The filter models asset prices using a linear state-space formulation with Gaussian process and observation noise.

The Kalman filter provides a smoothed estimate of the underlying price trend while preserving causality.
At each timestep, the filtered state depends only on information available up to that point, making it suitable for live trading and robust backtesting.

Short-term trend momentum is derived from changes in the Kalman-filtered signal and is used as the primary input to the portfolio allocation logic.

\section{Portfolio Strategy Design}

At each timestep, the strategy performs the following steps:
\begin{enumerate}
    \item Observe Kalman-filtered trend estimates for all assets
    \item Compute short-term trend momentum for each asset
    \item Allocate capital only to assets with positive trend momentum
    \item Distribute capital proportionally across positively trending assets
    \item Allocate remaining capital to Cash
\end{enumerate}

This rule-based allocation mechanism ensures that capital naturally shifts toward safer assets during weak or uncertain market conditions.
The logic is fully causal, interpretable, and avoids overfitting.

\section{Transaction Costs}

To discourage excessive rebalancing, a transaction cost of 0.1\% is applied to every change in portfolio allocation.
Transaction costs are calculated based on portfolio turnover at each timestep and are explicitly subtracted from portfolio returns.
This ensures realistic evaluation and meaningfully affects strategy behavior.

\section{Backtesting Framework}

The strategy is evaluated using a strictly causal backtesting framework:
\begin{itemize}
    \item Decisions at time $t$ use information only up to time $t$
    \item Trades are executed at time $t+1$
    \item No lookahead bias is introduced
\end{itemize}

The strategy is benchmarked against a buy-and-hold Nifty 50 portfolio to evaluate relative performance.

\section{Performance Metrics and Results}

The portfolio performance over the backtesting period is summarized using the following metrics:

\begin{itemize}
    \item \textbf{Annualized Return}: 0.63
    \item \textbf{Annualized Volatility}: 0.45
    \item \textbf{Sharpe Ratio}: 1.40
    \item \textbf{Maximum Drawdown}: -0.56
\end{itemize}

The results indicate strong risk-adjusted performance, with a Sharpe ratio greater than one, suggesting efficient compensation for risk taken.
While the strategy experiences periods of drawdown, capital allocation to cash helps limit prolonged downside exposure.
Overall performance is competitive when compared to a Nifty 50 buy-and-hold benchmark.

\section{Limitations and Future Improvements}

The current approach has several limitations.
Kalman filters assume linear dynamics and Gaussian noise, which financial markets often violate.
Trend-based signals may underperform in strongly mean-reverting environments.

Potential improvements include incorporating volatility-adjusted signals, regime detection mechanisms, reinforcement learning for adaptive allocation sizing, and more complex state-space models.

\section{Conclusion}

This project demonstrates how Kalman filtering can be extended from single-asset trend following to a multi-asset portfolio management system.
By combining causal filtering, disciplined allocation rules, and realistic transaction costs, the strategy achieves stable and interpretable performance suitable for real-world deployment.

\end{document}
