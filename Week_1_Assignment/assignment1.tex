\documentclass[11pt]{article}

\usepackage[a4paper,margin=1in]{geometry}
\usepackage{amsmath,amssymb}
\usepackage{graphicx}
\usepackage{booktabs}
\usepackage{hyperref}
\usepackage{setspace}
\usepackage{float}
\usepackage{listings}
\usepackage{xcolor}

\setstretch{1.1}

\lstset{
  language=Python,
  basicstyle=\ttfamily\small,
  keywordstyle=\color{blue},
  commentstyle=\color{gray},
  stringstyle=\color{red},
  numbers=left,
  numberstyle=\tiny,
  frame=single,
  breaklines=true,
  showstringspaces=false
}

\title{\textbf{WiDS Kalman Filtered Trend Trader}\\
Assignment 1}
\author{Lavanya Padole 24B1292}

\begin{document}
\maketitle

% ======================================================
\section{Question 1: Linear Regression}
% ======================================================

\subsection{Model Definition}

The multiple linear regression model with $p$ predictors is written as:
\[
y = X\beta + \epsilon,
\]
where $X \in \mathbb{R}^{n \times (p+1)}$ is the design matrix with an intercept column,
$\beta = (\beta_0, \beta_1, \ldots, \beta_p)^\top$ is the parameter vector, and
$\epsilon \sim \mathcal{N}(0, \sigma^2 I)$ is the noise term.

\subsection{OLS Objective and Solution}

Ordinary Least Squares minimizes the Mean Squared Error:
\[
J(\beta) = \frac{1}{n} \|X\beta - y\|^2
\]

Taking the gradient and setting it to zero yields:
\[
\hat{\beta} = (X^\top X)^{-1} X^\top y
\]

\subsection{Numerical Results}

The estimated OLS coefficients using NumPy are:
\[
\hat{\beta} =
\begin{bmatrix}
-0.075997 \\
3.019213 \\
-1.998648 \\
0.465321 \\
0.028746 \\
0.978880 \\
0.005197 \\
0.050117 \\
-0.917746 \\
1.951978 \\
-0.012285 \\
0.057589 \\
0.027415
\end{bmatrix}
\]

These coefficients exactly match those obtained using
\texttt{sklearn.linear\_model.LinearRegression}, confirming the correctness of the
closed-form OLS implementation.

\subsection{Residual Diagnostics}

Residuals versus fitted values and a Q--Q plot were analyzed.
The residuals show no strong systematic pattern, suggesting approximate homoscedasticity.
The Q--Q plot indicates mild deviations from normality in the tails, which is common
in practical regression problems but does not severely violate model assumptions.

\subsection{Leverage and Influence}

The hat matrix was computed as:
\[
H = X (X^\top X)^{-1} X^\top
\]

The top ten high-leverage observations were:
\[
\{212, 168, 241, 124, 39, 293, 112, 240, 160, 96\}
\]

Cook’s distance identified the following influential points:
\[
\{155, 296, 267, 59, 286, 117, 86, 146, 293, 116\}
\]

These observations have the greatest potential impact on regression coefficients.
Not all high-leverage points are influential, highlighting the importance of considering
both leverage and residual magnitude.

% ======================================================
\section{Question 2: Salary Prediction and Bias Detection}
% ======================================================

\subsection{Exploratory Data Analysis}

The dataset contains 12,000 employee records.
The mean salary is \$105,862 with a standard deviation of \$28,251, indicating substantial
salary dispersion and the presence of high-income outliers.
Salary distributions differ across gender categories as observed from boxplots.

\subsection{Preprocessing}

Missing values were removed to ensure numerical stability.
Categorical features were encoded using one-hot encoding with a reference category dropped
to avoid multicollinearity.
A stratified train-test split was performed using gender to preserve group proportions.

\subsection{Model Training and Evaluation}

An OLS Linear Regression model was trained.

\begin{itemize}
\item RMSE: \textbf{17,582.01}
\item MAE: \textbf{7,465.70}
\item $R^2$: \textbf{0.629}
\end{itemize}

The model explains approximately 63\% of the variance in employee salaries.

\subsection{Fairness Analysis}

Gender was treated as the protected attribute.

\begin{center}
\begin{tabular}{lcc}
\toprule
Metric & Male vs Female & Male vs Other \\
\midrule
Mean Prediction Difference & 5,225.21 & 1,405.19 \\
MAE Difference & -179.18 & 966.34 \\
\bottomrule
\end{tabular}
\end{center}

Male employees receive higher predicted salaries on average.
However, the MAE difference between male and female employees is small, suggesting similar
prediction accuracy across groups.

\subsection{Residual Bias and Statistical Testing}

A two-sample t-test on residuals between male and female groups produced a p-value of 0.064.
At the 5\% significance level, this result indicates no statistically significant difference
in mean residuals, suggesting the model does not systematically overestimate or underestimate
salary for either group.

% ======================================================
\section{Question 3: Deep Neural Network Classifier}
% ======================================================

\subsection{Model Architecture}

A deep neural network named \texttt{DigitClassifier} was designed with the following structure:
\begin{itemize}
\item Input layer of dimension 784
\item Hidden layer 1: 256 neurons with ReLU activation
\item Hidden layer 2: 128 neurons with ReLU activation
\item Output layer: 10 logits corresponding to digit classes
\end{itemize}

\subsection{Training Setup}

The model is trained for five epochs using the Adam optimizer with a learning rate of 0.001
and CrossEntropyLoss. A batch size of 64 is used. The training loop consists of forward pass,
loss computation, backward propagation, and optimizer updates.

\subsection{Why ReLU is Preferred}

ReLU is preferred over Sigmoid and Tanh for deep networks for two primary reasons:
\begin{itemize}
\item ReLU mitigates the vanishing gradient problem by maintaining non-saturating gradients
for positive inputs.
\item It is computationally efficient, enabling faster convergence during training.
\end{itemize}

\subsection{Role of Autograd}

PyTorch’s autograd engine automatically tracks tensor operations to construct a dynamic
computational graph during the forward pass.
During backpropagation, autograd applies the chain rule to compute gradients of the loss
with respect to model parameters, enabling efficient and flexible gradient-based optimization.

\subsection{Implementation Note}

The PyTorch implementation is provided as requested. Model execution was not performed,
as the assignment assumes the availability of predefined data loaders.

\end{document}
