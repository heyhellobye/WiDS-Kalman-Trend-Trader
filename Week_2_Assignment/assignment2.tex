\documentclass[11pt]{article}

\usepackage[a4paper,margin=1in]{geometry}
\usepackage{amsmath,amssymb}
\usepackage{graphicx}
\usepackage{booktabs}
\usepackage{hyperref}
\usepackage{setspace}
\usepackage{float}

\setstretch{1.1}

\title{\textbf{WiDS Kalman Filtered Trend Trader}\\
Assignment 2}
\author{Lavanya Padole 24B1292}

\begin{document}
\maketitle

% ======================================================
\section{Introduction}
% ======================================================

Financial markets exhibit non-stationary dynamics, where relationships between prices,
returns, volatility, and market features evolve over time. Static models are often unable
to adapt to such changes, leading to degraded performance.

In this assignment, a systematic trading strategy for Microsoft Corporation (MSFT) is
developed using Kalman Filters to model time-varying latent relationships and a supervised
machine learning model to generate predictive trading signals. The strategy is evaluated
using a causal backtesting framework and compared to a buy-and-hold benchmark.

% ======================================================
\section{Data Collection}
% ======================================================

Daily historical market data for Microsoft Corp. (MSFT) from 2015 to 2024 was obtained from
Yahoo Finance. Adjusted closing prices and trading volume were used to ensure consistency
across corporate actions.

All preprocessing steps maintain strict causality, ensuring that no future information
is used in model estimation or trading decisions.

% ======================================================
\section{Feature Engineering}
% ======================================================

A diverse set of market features was constructed to capture price dynamics, momentum, and
risk characteristics:

\begin{itemize}
\item Log returns and lagged returns
\item Moving averages (5-day, 20-day, 60-day)
\item Rate of Change (ROC)
\item Rolling volatility measures
\item Volume-based indicators
\end{itemize}

These features provide complementary information about short-term momentum, long-term
trend structure, and market uncertainty.

% ======================================================
\section{Kalman Filter Model}
% ======================================================

A state-space model was formulated in which the latent state represents time-varying
regression coefficients linking engineered features to MSFT returns.

\subsection{Model Formulation}

The observation equation is defined as:
\[
y_t = X_t \beta_t + \epsilon_t, \quad \epsilon_t \sim \mathcal{N}(0, R)
\]

The state transition equation is:
\[
\beta_t = \beta_{t-1} + \eta_t, \quad \eta_t \sim \mathcal{N}(0, Q)
\]

Here, $\beta_t$ denotes the latent parameter vector, while $Q$ and $R$ control parameter drift
and observation noise respectively.

The Kalman Filter recursively estimates $\beta_t$ using only current and past information,
allowing the model to adapt to evolving market conditions while remaining causal.

\subsection{Filtered Parameters}

Kalman-filtered parameters exhibit smooth temporal evolution, reflecting gradual changes in
the relevance of different market features across regimes.

% ======================================================
\section{Machine Learning Integration}
% ======================================================

The Kalman-filtered latent parameters were used as inputs to a supervised learning model
to predict the next-period price ratio of MSFT.

A linear regression model was selected due to its interpretability, stability, and low risk
of overfitting when applied to time-series data with evolving dynamics. Predictions are made
using only information available at the current timestep.

% ======================================================
\section{Trading Strategy Design}
% ======================================================

Trading signals are derived from the predicted price ratio:

\begin{itemize}
\item \textbf{Buy Signal}: Predicted ratio exceeds the current ratio by a predefined threshold
\item \textbf{Sell Signal}: Predicted ratio falls below the current ratio by a predefined threshold
\end{itemize}

Risk management constraints include:
\begin{itemize}
\item Long/short positioning without leverage
\item Transaction cost adjustment
\item Position updates executed at $t+1$
\end{itemize}

All signals are generated causally and are based solely on present and historical data.

% ======================================================
\section{Backtesting Framework}
% ======================================================

The strategy was evaluated using a walk-forward backtesting framework:

\begin{itemize}
\item Signals generated at time $t$ are executed at $t+1$
\item Daily profit and loss (PnL) is computed based on position returns
\item Transaction costs are explicitly incorporated
\end{itemize}

A buy-and-hold MSFT strategy is used as a benchmark for comparison.

% ======================================================
\section{Performance Evaluation}
% ======================================================

The trading strategy was evaluated using multiple quantitative metrics:

\begin{itemize}
\item \textbf{Cumulative Return}: 1.124
\item \textbf{Sharpe Ratio}: 0.60
\item \textbf{Maximum Drawdown}: -0.16
\item \textbf{Win/Loss Ratio}: 0.112
\end{itemize}

The cumulative return indicates that the strategy more than doubled the initial capital
over the evaluation period. A Sharpe ratio of 0.60 reflects moderate risk-adjusted performance,
consistent with systematic trend-following approaches.

The maximum drawdown of approximately 16\% suggests controlled downside risk relative to
typical equity market behavior. The low win/loss ratio implies that the strategy relies on
fewer, larger profitable trades rather than frequent small gains.

When compared to a buy-and-hold MSFT strategy, the proposed approach offers improved drawdown
control while maintaining competitive long-term returns.

% ======================================================
\section{Discussion and Limitations}
% ======================================================

The results demonstrate the effectiveness of combining Kalman-filtered time-varying parameter
estimation with supervised learning in non-stationary financial environments.

However, several limitations remain:
\begin{itemize}
\item Linear modeling assumptions may limit expressiveness
\item Sensitivity to noisy short-term signals
\item Dependence on feature selection and threshold parameters
\end{itemize}

Future improvements may include nonlinear models, regime detection, or adaptive noise
estimation within the Kalman framework.

% ======================================================
\section{Conclusion}
% ======================================================

This assignment presents a complete end-to-end trading strategy using Kalman Filters and
machine learning. By modeling evolving market relationships and enforcing causal trading
rules, the strategy adapts to non-stationary market behavior and demonstrates stable
performance relative to static benchmarks.

\end{document}D